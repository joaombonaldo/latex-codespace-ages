% Espaço vertical:
\vspace{1cm}

% Negrito:
\textbf{texto}

% itálico
\textit{italico}

% Underline
\underline{sublinhado}

% Alinhado à direita:
\begin{flushright}
  direita
\end{flushright}

% Alinhado à esquerda:
\begin{flushleft}
  esquerda
\end{flushleft}

% Figuras:
% \begin{figure}[h!]
%   \label{figDemo}
%   \centering
%   \includegraphics*[width=10cm]{imagefile}
%   \caption[short]{Título que vai aparecer la}
% \end{figure}

% Figura nos padrões do relatório da AGES
\begin{figure}[H] % H é para forçar o posicionamento REAL
    \label{figDemo}
    \centering
    \small
    \caption{Modelo do Banco de Dados}
    \includegraphics[width=1\linewidth]{imagefile}
    Fonte: XXXX
\end{figure}

% Aí pode utilizar como referencia
`Adorei a figura\ref{figDemo}`

% Listas (bullet)
\begin{itemize}
  \item Teste 1
  \item Teste 2
\end{itemize}

% Lista (enumerada)
\begin{enumerate}
  \item Teste 1
  \item Teste 2
\end{enumerate}

% itens matemáticos, usar $ para englobar.
% ex:
$a \rightarrow t + (1 \times 3)$

% Tamanhos de texto
\tiny texto
\small texto
\large texto
\huge texto
\scriptsize texto
\footnotesize texto

% Quote
\begin{quotation}
  texto aqui\ldots
\end{quotation}

% Citação:
% Na bibliografia.bib, coloque: @book ou @article, @misc

%  aí ja pode utilizar com o \cite
% \bibliographystyle{plainnat}
Teste\cite{chave}
% Com página
Teste\cite[capítulo, p.~215]{chave}
% Somente o ano
Teste\citeyear{chave}
