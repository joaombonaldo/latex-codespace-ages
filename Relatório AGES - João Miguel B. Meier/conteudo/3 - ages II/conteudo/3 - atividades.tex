\section[Atividades Desempenhadas Pelo Aluno no Projeto]{Atividades Desempenhadas Pelo Aluno no Projeto}

\subsection{Sprint 0}

O projeto "Let Me Trial" começou imediatamente após uma reunião inicial comos stakeholders, que apresentaram a visão geral do projeto e as expectativas em relação ao sistema. Durante esta reunião, foi identificada a necessidade de um sistema web que facilitasse a conexão, ou "match", entre pacientes de médicos e estudos clínicos adequados. Após a apresentação, discutimos as tecnologias que seriam utilizadas e as áreas específicas de interesse de cada membro da equipe, decidindo entre backend, frontend ou fullstack.\\


Com as diretrizes claras, a equipe passou a definir as tarefas específicas para
a Sprint 0. Decidimos que as primeiras entregas seriam os fluxogramas do processo de funcionamento do software e os mockups das interfaces de usuário, utilizando o Figma como ferramenta devido à sua flexibilidade e recursos de colaboração. Cada membro da equipe desenvolveu inicialmente um fluxograma baseado na sua interpretação do sistema, que posteriormente foi discutido em grupo para consolidar um fluxograma oficial que guiaria o desenvolvimento.\\

Quanto aos mockups, distribuímos as responsabilidades de cada tela entre os membros da equipe. João, especificamente, ficou encarregado do mockup da tela de Login. Essa atividade foi essencial para visualizar a interface do usuário e garantir uma experiência consistente e intuitiva. Após a conclusão dos mockups, a equipe também discutiu e definiu um padrão de cores e estilo das interfaces, o que ajudou a manter a coesão visual do projeto.\\

Além das tarefas de design, como AGES I, foi dedicado um tempo considerável ao estudo do Java SpringBoot, a tecnologia escolhida para o desenvolvimento do backend. Realizei pequenos projetos para compreender melhor as funcionalidades da biblioteca e explorar aspectos da arquitetura de sistemas. Para reforçar meu entendimento, também revisei algumas aulas de Gerenciamento e Configuração de Software, focando em conceitos de Git, essenciais para o trabalho em equipe e controle de versões.\\

Durante esse período, enfrentei desafios adicionais propostos pelo AGES IV, incluindo um exercício prático de backend. Inicialmente, encontrei dificuldades devido à complexidade das tarefas que estavam além da minha base de conhecimento naquele momento. No entanto, com a assistência contínua dos AGES IV, através de encontros virtuais nos finais de semana, consegui superar essas barreiras e completar o desafio proposto.\\

Essa primeira Sprint foi crucial não apenas para definir a estrutura e o design do projeto, mas também para o desenvolvimento pessoal e técnico de cada membro da equipe, proporcionando uma base sólida para as próximas fases do projeto.

O projeto "Dashboard para a Polícia Civil" teve início com a Sprint 0, onde a equipe se reuniu para alinhar a visão geral do projeto e definir as diretrizes para as atividades subsequentes. A primeira tarefa da Sprint foi a criação dos protótipos no Figma e a modelagem do Banco de Dados utilizando o Astah, ferramentas essenciais para o desenvolvimento inicial do projeto. Durante a Sprint, a equipe trabalhou para definir as interfaces de usuário e a estrutura de dados, garantindo que as necessidades da Polícia Civil fossem atendidas de forma eficiente.\\
Como AGES II, fiquei encarregado da modelagem do banco de dados, em colaboração com os outros AGES II. Juntos, trabalhamos na definição das tabelas e seus relacionamentos no Astah, o que ajudou a garantir que a estrutura de dados fosse eficiente e escalável para o uso no dashboard. Além disso, devido à minha experiência na modelagem do banco de dados, pude contribuir com minha experiência para a criação de algumas telas no Figma, focadas nos gráficos de dashboards. Essas telas foram projetadas com base nas planilhas que seriam integradas ao sistema, o que me proporcionou um entendimento mais profundo sobre quais tipos de gráficos seriam mais apropriados para a aplicação e como visualizá-los de forma clara e eficiente.\\
Durante a execução dessa Sprint, não foram encontrados problemas significativos. As atividades previstas foram completadas dentro do prazo, sem obstáculos ou dificuldades técnicas. A equipe conseguiu cumprir todos os objetivos propostos, refletindo um bom alinhamento entre os membros, o que reflete uma boa organização e alinhamento entre os membros.

\subsection{Sprint 1}

Após a apresentação dos resultados da Sprint 0 aos stakeholders, onde mostramos os fluxogramas e mockups desenvolvidos, e onde discutimos o planejamento para a Sprint 1, nossa equipe foi organizada em quatro squads. Cada squad foi composta por uma divisão equilibrada entre membros focados em backend e frontend. Eu fui alocado na squad 4, que recebeu a responsabilidade pela User Story 03, dedicada ao cadastro de pacientes.\\

Essa user story especificamente envolvia o desenvolvimento de funcionalidades para cadastrar pacientes sob a supervisão de um médico. Como ainda não havíamos implementado a funcionalidade de cadastro de médicos, utilizamos um médico de teste para integrar essa nova funcionalidade. Junto com Guilherme Ochoa, também do backend da squad 4, decidimos adotar uma abordagem colaborativa em vez de dividir tarefas especificamente. Optamos por trabalhar juntos em todos os aspectos da funcionalidade, o que nos permitiu aprender e complementar um ao outro de maneira mais eficiente.\\

Inicialmente, enfrentei desafios significativos devido à falta de experiência prática com projetos reais. Foi durante esse período que foi percebido a importância do Pair Programming, uma técnica de programação colaborativa que se provou fundamental para o meu desenvolvimento. Nas sessões de Pair Programming, realizadas virtualmente, recebi orientação intensiva do AGES IV sobre a arquitetura hexagonal utilizada no backend. Essa orientação foi crucial para entender onde e como implementar cada segmento da funcionalidade em desenvolvimento.\\

Focamos inicialmente na criação de funções robustas para a verificação de dados e tratamento de exceções. Após estabelecermos uma base sólida para a funcionalidade de cadastro, prosseguimos para a etapa de testes.\\

Em colaboração com o AGES IV, desenvolvemos testes unitários utilizando a biblioteca Mockito. A experiência de desenvolver testes unitários foi enriquecedora, proporcionando uma compreensão mais profunda das práticas de desenvolvimento de software e da importância de uma base de código testável e sustentável.\\

A Sprint 1 provou ser uma experiência intensiva e instrutiva, consolidando as habilidades técnicas da equipe e aprimorando nossa capacidade de trabalhar de forma colaborativa sob a estrutura de desenvolvimento ágil. As lições aprendidas e as habilidades desenvolvidas durante essa fase do projeto serão fundamentais para o sucesso das próximas sprints.
O projeto "Dashboard para a Polícia Civil" avançou para a Sprint 1, onde a equipe foi dividida em duas áreas de atuação: frontend e backend. Cada membro foi designado para sua respectiva área, e como eu fui designado para o backend, a minha principal tarefa na Sprint 1 foi a criação de uma rota para receber arquivos CSV, além de realizar alterações no banco de dados conforme a necessidade para atender aos requisitos do sistema. A equipe estava focada na integração de dados externos através de arquivos CSV, o que exigia uma lógica robusta para garantir a integridade e a consistência desses dados dentro da aplicação.\\
Durante a Sprint 1, consegui concluir a criação da rota para receber arquivos CSV, bem como implementar a lógica para ler o arquivo e validá-lo de forma eficiente. Uma parte importante desse processo foi a implementação da lógica para direcionar cada tipo de arquivo para a função responsável por sua validação. Para garantir que o código fosse flexível e fácil de expandir no futuro, utilizei o design pattern 'strategy'. Com esse padrão, a lógica foi modularizada, permitindo que novos tipos de validação fossem facilmente adicionados sem a necessidade de modificar o código existente.\\
Além dessas atividades, também comecei a implementação de uma possível pipeline para o projeto de backend-file, com foco na automação de processos de integração contínua e entrega contínua (CI/CD) utilizando GitLab. Embora essa implementação ainda esteja em seus estágios iniciais, ela representa um avanço importante para garantir a escalabilidade e a manutenção do projeto.\\
Durante a execução da Sprint 1, enfrentei alguns problemas para rodar a aplicação de infraestrutura no meu computador pelo Docker, o que causou algum atraso no início das atividades. No entanto, esse problema não afetou significativamente minha capacidade de concluir as tarefas previstas, e com a ajuda dos colegas, consegui superar esse obstáculo rapidamente.

\subsection{Sprint 2}

A Sprint 2 começou com o foco em desenvolver e testar a funcionalidade de visualizar pacientes cadastrados sob a supervisão de um médico específico. Esta User Story foi alocada para nossa squad 4, continuando a parceria entre mim e o Guilherme Ochoa no backend. Concluímos esta funcionalidade com sucesso, seguindo o planejamento inicial e as técnicas já consolidadas de desenvolvimento ágil que havíamos adotado.\\

A implementação dessa funcionalidade envolveu a criação de endpoints específicos na API, garantindo a comunicação eficiente e segura entre o frontend e o backend. A experiência adquirida na sprint anterior, especialmente em termos de testes unitários e arquitetura de software, facilitou significativamente nossa tarefa. Com a utilização da biblioteca Mockito, desenvolvemos testes que validaram a integridade e funcionalidade dos novos endpoints de forma eficaz.\\
No entanto, a segunda parte da sprint envolveu um desafio maior com a User Story para cadastrar áreas na API administrador. A dificuldade aqui foi a necessidade de construir essa funcionalidade do zero, uma vez que a API administrador até então possuía apenas o código base sem funcionalidades específicas implementadas. Esse trabalho revelou-se mais complicado do que o antecipado, e não conseguimos finalizar a funcionalidade dentro do prazo desta sprint.\\

Outro ponto desafiador que acabou ocorrendo nessa Sprint 2, foram as enchentes que afetaram a cidade de Porto Alegre e Região Metropolitana. Com o cancelamento das aulas por duas semanas, o grupo inteiro acabou se desconectando um pouco do projeto, inclusive, acabamos perdendo alguns colegas por desistência da disciplina, o que, também, acabou desfalcando o time como um todo.\\
O reforço nos aprendizados sobre testes unitários foi um ponto positivo, visto que acabei fazendo mais testes para este endpoint de visualizar pacientes cadastrados sob a supervisão de um médico específico.\\

Para as próximas etapas do projeto, planejamos continuar o desenvolvimento da User Story de cadastrar áreas na API administrador, além de iniciar a integração CI/CD para automatizar os processos de teste e deployment. Este próximo passo é crucial para manter a consistência e a qualidade do software, permitindo uma integração contínua e entrega contínua das funcionalidades desenvolvidas.
\subsection{Sprint 3}
Durante a Sprint 3, nosso foco estava em concluir a implementação da User Story para cadastrar áreas na API Administrador e avançar nos estudos de CI/CD, com ênfase especial no Kubernetes. Esta etapa do projeto se mostrou desafiadora e instrutiva, proporcionando uma oportunidade significativa para desenvolver habilidades técnicas avançadas e aplicá-las no projeto.\\

A User Story de cadastrar áreas na API Administrador, uma continuação do trabalho iniciado na sprint anterior, não foi completada a 100 por cento. Enfrentamos dificuldades significativas devido à complexidade de implementar essa funcionalidade do zero, o que exigiu uma compreensão profunda das necessidades do sistema e uma integração eficaz com as partes já existentes da API. Apesar dos desafios, fizemos progressos consideráveis, estabelecendo uma base sólida sobre a qual poderíamos construir na conclusão do projeto.\\

Paralelamente, dediquei um tempo substancial para estudar Kubernetes, uma ferramenta essencial para a gestão de containers e automatização de CI/CD. O aprendizado adquirido foi profundamente enriquecedor, aumentando minha compreensão sobre como essas tecnologias podem ser aplicadas para melhorar a eficiência e escalabilidade das aplicações. Esse estudo não apenas ampliou meu conhecimento técnico mas também me permitiu auxiliar efetivamente outros membros da equipe.\\

Além da minha carga de trabalho principal, auxiliei o Matheus Bueno na task de dar o ‘Match’ entre os estudos e os pacientes, contribuindo na elaboração de testes e na revisão do código para o Merge Request. Esta colaboração foi uma oportunidade valiosa para aplicar as habilidades em testes que havia desenvolvido até o momento, além de fortalecer a dinâmica de equipe e o compartilhamento de conhecimentos.\\
As lições aprendidas durante esta sprint reforçaram a importância de uma base sólida em testes e a compreensão de ferramentas de CI/CD como Kubernetes. Estes
aprendizados serão cruciais não apenas para as tarefas técnicas futuras, mas também para a minha carreira como um todo.\\
Para a próxima sprint, que será a última deste projeto, foi dada prioridade a algumas tarefas específicas. Fiquei responsável por desenvolver a funcionalidade de visualizar todas as respostas do paciente no backend.

\subsection{Sprint 4}

A Sprint 4 marcou a fase final do nosso projeto, focada em desenvolver a rota para buscar todas as respostas do paciente no backend. Este objetivo representava um componente essencial para a funcionalidade de nossa API, permitindo que os dados dos pacientes fossem acessados de forma organizada e eficiente.\\

A tarefa principal da sprint foi implementar essa rota do zero. Apesar de inicialmente enfrentar dificuldades técnicas, especialmente na construção de uma base sólida para a rota, a colaboração com o AGES IV, Jhonata, no início do processo foi fundamental. Sua ajuda permitiu superar os obstáculos iniciais e me deu o impulso necessário para prosseguir de forma independente. Uma vez superados esses desafios iniciais, consegui desenvolver o restante da funcionalidade com maior confiança e eficiência.\\

Além de concluir a rota principal, dediquei parte do meu tempo para ampliar a cobertura de testes unitários, focando em algumas funcionalidades da classe ‘RespostaService’ que anteriormente estavam sem testes.\\

Um contratempo adicional durante esta sprint foi um problema técnico com meu computador, o que me obrigou a adaptar rapidamente meu plano de trabalho. Acabei que todas as tarefas que fui fazer, tinha que pedir para algum outro colega compartilhar o código comigo ao vivo, via uma extensão do VSCode, chamada ‘LiveShare’.\\

Em termos de aprendizado técnico, esta sprint foi extremamente produtiva. Aprofundei significativamente meu conhecimento em arquitetura de software e em testes unitários. Além disso, aprendi a utilizar Data Transfer Objects (DTOs) corretamente, uma habilidade crucial para a manipulação eficiente de dados entre diferentes camadas de uma aplicação.\\

Como sendo a última Sprint do projeto, sinto que tive um ótimo progresso em conhecimento técnico em desenvolvimento de software. Com certeza, estou saindo desta última Sprint com muito mais conhecimento e confiança para enfrentar outros projeto grandes, sejam pessoais ou dentro de empresas.
