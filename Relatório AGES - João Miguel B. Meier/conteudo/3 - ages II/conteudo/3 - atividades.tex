\section[Atividades Desempenhadas Pelo Aluno no Projeto]{Atividades Desempenhadas Pelo Aluno no Projeto}

\subsection{Sprint 0}

No mínimo uma página contendo tudo que o aluno fez na Sprint 0.
O projeto "Dashboard para a Polícia Civil" teve início com a Sprint 0, onde a equipe se reuniu para alinhar a visão geral do projeto e definir as diretrizes para as atividades subsequentes. A primeira tarefa da Sprint foi a criação dos protótipos no Figma e a modelagem do Banco de Dados utilizando o Astah, ferramentas essenciais para o desenvolvimento inicial do projeto. Durante a Sprint, a equipe trabalhou para definir as interfaces de usuário e a estrutura de dados, garantindo que as necessidades da Polícia Civil fossem atendidas de forma eficiente.\\
Como AGES II, fiquei encarregado da modelagem do banco de dados, em colaboração com os outros AGES II. Juntos, trabalhamos na definição das tabelas e seus relacionamentos no Astah, o que ajudou a garantir que a estrutura de dados fosse eficiente e escalável para o uso no dashboard. Além disso, devido à minha experiência na modelagem do banco de dados, pude contribuir com minha experiência para a criação de algumas telas no Figma, focadas nos gráficos de dashboards. Essas telas foram projetadas com base nas planilhas que seriam integradas ao sistema, o que me proporcionou um entendimento mais profundo sobre quais tipos de gráficos seriam mais apropriados para a aplicação e como visualizá-los de forma clara e eficiente.\\
Durante a execução dessa Sprint, não foram encontrados problemas significativos. As atividades previstas foram completadas dentro do prazo, sem obstáculos ou dificuldades técnicas. A equipe conseguiu cumprir todos os objetivos propostos, refletindo um bom alinhamento entre os membros, o que reflete uma boa organização e alinhamento entre os membros.

\subsection{Sprint 1}

Após a apresentação dos resultados da Sprint 0 aos stakeholders, onde mostramos os fluxogramas e mockups desenvolvidos, e onde discutimos o planejamento para a Sprint 1, nossa equipe foi organizada em quatro squads. Cada squad foi composta por uma divisão equilibrada entre membros focados em backend e frontend. Eu fui alocado na squad 4, que recebeu a responsabilidade pela User Story 03, dedicada ao cadastro de pacientes.\\

Essa user story especificamente envolvia o desenvolvimento de funcionalidades para cadastrar pacientes sob a supervisão de um médico. Como ainda não havíamos implementado a funcionalidade de cadastro de médicos, utilizamos um médico de teste para integrar essa nova funcionalidade. Junto com Guilherme Ochoa, também do backend da squad 4, decidimos adotar uma abordagem colaborativa em vez de dividir tarefas especificamente. Optamos por trabalhar juntos em todos os aspectos da funcionalidade, o que nos permitiu aprender e complementar um ao outro de maneira mais eficiente.\\

Inicialmente, enfrentei desafios significativos devido à falta de experiência prática com projetos reais. Foi durante esse período que foi percebido a importância do Pair Programming, uma técnica de programação colaborativa que se provou fundamental para o meu desenvolvimento. Nas sessões de Pair Programming, realizadas virtualmente, recebi orientação intensiva do AGES IV sobre a arquitetura hexagonal utilizada no backend. Essa orientação foi crucial para entender onde e como implementar cada segmento da funcionalidade em desenvolvimento.\\

Focamos inicialmente na criação de funções robustas para a verificação de dados e tratamento de exceções. Após estabelecermos uma base sólida para a funcionalidade de cadastro, prosseguimos para a etapa de testes.\\

Em colaboração com o AGES IV, desenvolvemos testes unitários utilizando a biblioteca Mockito. A experiência de desenvolver testes unitários foi enriquecedora, proporcionando uma compreensão mais profunda das práticas de desenvolvimento de software e da importância de uma base de código testável e sustentável.\\

A Sprint 1 provou ser uma experiência intensiva e instrutiva, consolidando as habilidades técnicas da equipe e aprimorando nossa capacidade de trabalhar de forma colaborativa sob a estrutura de desenvolvimento ágil. As lições aprendidas e as habilidades desenvolvidas durante essa fase do projeto serão fundamentais para o sucesso das próximas sprints.
O projeto "Dashboard para a Polícia Civil" avançou para a Sprint 1, onde a equipe foi dividida em duas áreas de atuação: frontend e backend. Cada membro foi designado para sua respectiva área, e como eu fui designado para o backend, a minha principal tarefa na Sprint 1 foi a criação de uma rota para receber arquivos CSV, além de realizar alterações no banco de dados conforme a necessidade para atender aos requisitos do sistema. A equipe estava focada na integração de dados externos através de arquivos CSV, o que exigia uma lógica robusta para garantir a integridade e a consistência desses dados dentro da aplicação.\\
Durante a Sprint 1, consegui concluir a criação da rota para receber arquivos CSV, bem como implementar a lógica para ler o arquivo e validá-lo de forma eficiente. Uma parte importante desse processo foi a implementação da lógica para direcionar cada tipo de arquivo para a função responsável por sua validação. Para garantir que o código fosse flexível e fácil de expandir no futuro, utilizei o design pattern 'strategy'. Com esse padrão, a lógica foi modularizada, permitindo que novos tipos de validação fossem facilmente adicionados sem a necessidade de modificar o código existente.\\
Além dessas atividades, também comecei a implementação de uma possível pipeline para o projeto de backend-file, com foco na automação de processos de integração contínua e entrega contínua (CI/CD) utilizando GitLab. Embora essa implementação ainda esteja em seus estágios iniciais, ela representa um avanço importante para garantir a escalabilidade e a manutenção do projeto.\\
Durante a execução da Sprint 1, enfrentei alguns problemas para rodar a aplicação de infraestrutura no meu computador pelo Docker, o que causou algum atraso no início das atividades. No entanto, esse problema não afetou significativamente minha capacidade de concluir as tarefas previstas, e com a ajuda dos colegas, consegui superar esse obstáculo rapidamente.

\subsection{Sprint 2}

No mínimo uma página contendo tudo que o aluno fez na Sprint 2.
\subsection{Sprint 3}
No mínimo uma página contendo tudo que o aluno fez na Sprint 3.
\subsection{Sprint 4}

No mínimo uma página contendo tudo que o aluno fez na Sprint 4.