\chapter[APRESENTAÇÃO DA TRAJETÓRIA DO ALUNO]{APRESENTAÇÃO DA TRAJETÓRIA DO ALUNO}

Desde 2023, João Miguel Bonaldo Meier é graduando em Engenharia de Software. Seu primeiro contato com programação ocorreu em 2017, durante um curso técnico em informática. Já na universidade, em maio de 2023, teve sua iniciação acadêmica no PET Informática, onde se envolveu com tarefas de pesquisa científica e extensão. No mesmo ano, conquistou uma vaga de estágio em Automação Fiscal na Dell Technologies. Durante esse período, adquiriu conhecimentos em tecnologias como Python, SQL e Power BI, além de conceitos financeiros e desenvolveu habilidades interpessoais, como pensamento organizacional, responsabilidade, comunicação, liderança e trabalho em equipe.

Em 2024, João iniciou a AGES I, uma prática integrada ao seu curso, com o projeto "LET ME TRIAL". Este desafio o colocou frente a conceitos complexos de software que vão além do ensino formal até o terceiro semestre. Sem um amplo conhecimento técnico prévio, além do obtido em disciplinas como Algoritmos e Estruturas de Dados, Programação Orientada a Objetos, e Gerenciamento e Configuração de Software, João vem desenvolvendo habilidades técnicas avançadas em Java SpringBoot, bibliotecas de testes unitários, como Mockito, além de outras ferramentas usadas em todo o processo de desenvolvimento de software, como: Docker, Figma, e conceitos de arquitetura de software.

Em maio de 2024, João mudou para outro time dentro da Dell Technologies, passando a atuar como Estagiário de Engenharia de Software. Neste novo papel, focou em desenvolvimento Salesforce (Apex), além de trabalhar com HTML, CSS, JavaScript e o framework Lightning Web Components (LWC). Além disso, ganhou experiência com GitLab CI/CD, pipelines e scripts Shell, ao construir uma pipeline de deploy para seu time de Salesforce.


Utilize referências também, modificando o arquivo \textit{bibliografia/bibliografia.bib} e citando-os com o comando\cite{artigo}.
    