\section[Introdução]{Introdução}

O projeto "Let Me Trial" foi concebido para atender a uma lacuna identificada no mercado de medicina pelo médico Giovani Gadonski e seu sócio Matheus Stortti. Eles observaram uma carência de softwares que facilitam o encaminhamento de pacientes para estudos clínicos. O "Let Me Trial" é uma plataforma web destinada a médicos que desejam encontrar o estudo clínico mais adequado para seus pacientes. Utilizando o software, o médico cadastra informações pessoais do paciente, e o sistema automaticamente sugere estudos clínicos compatíveis. Após escolher um estudo, o médico responde a um questionário complementar específico para aquele estudo. Ao final, o software indica se o paciente atende aos requisitos para participar do processo de seleção do estudo clínico ou se ele não possui os critérios necessários.

O projeto foi realizado durante o primeiro semestre de 2024, especificamente de 1º de março a 21 de junho. As reuniões da equipe aconteceram presencialmente todas as sextas-feiras, das 19h15 às 22h30, exceto em feriados e, nos períodos de aulas remotas durante a enchente que afetou a cidade de Porto Alegre no mês de mario. Os encontros com os Stakeholders eram realizados de maneira presencial ou remota, via a plataforma Zoom, todas as finais de Sprint, para ser feito a Sprint Demo, como conhecida no mundo de desenvolvimento ágil. A duração detalhada de cada Sprint é possível visualizar na tabela abaixo:

\begin{table}[H]
    \centering
    \caption{Período de cada Sprint - AGES I}
    \begin{tabular}{|c|c|}
        \hline
        \textbf{Sprints} & \textbf{Período} \\
        \hline
        0 & 08/03/2024 a 22/03/2024\\
        1 & 22/03/2024 a 19/04/2024 \\
        2 & 119/04/2024 a 10/05/2024 \\
        3 & 10/05/2024 a 31/05/2024 \\
        4 & 31/05/2024 a 21/06/2024 \\
        \hline
    \end{tabular}
\end{table}

Este projeto foi desenvolvimento sob a orientação do professor Jorge Horácio. A equipe responsável pelo projeto ‘Let me Trial’ está retratada na Figura 1.

\begin{figure}[H]
    \centering
    \small
    \caption{Time responsável pelo projeto}
    \includegraphics[width=1\linewidth]{conteudo/2 - ages I/conteudo/figures/foto-letmetrial.png}
    \textit{Fonte: Wiki do Projeto}
    \label{fig:projeto-time}
\end{figure}