\section[Desenvolvimento do Projeto]{Desenvolvimento do Projeto}



\subsection*{Repositório do Código Fonte do Projeto}
    Um repositório público foi estabelecido no GitLab da AGES, abrangendo sete componentes distintos: api-auth, api-medico, infraestrutura, web-administrador, web- médico, web-site e uma seção dedicada à documentação no formato de Wiki.

    Para mais detalhes, consulte este link:
    {\url{https://tools.ages.pucrs.br/let-me-trial}}.


\subsection{Banco de Dados Utilizado}
  O banco de dados escolhido para o projeto foi do tipo relacional. Utilizamos PostgreSQL devido aos recursos avançados, como transações ACID (Atomicidade, Consistência, Isolamento e Durabilidade). Na \texttt{\href{http://example.com}{wiki}} do repositório, há uma seção dedicada ao banco de dados (BD), onde está detalhada toda a estrutura utilizada no projeto. Você pode visualizar o modelo conceitual completo do banco de dados na figura abaixo.


\subsection{Arquitetura Utilizada}
  Como arquitetura para o projeto, foi escolhida a arquitetura hexagonal. A arquitetura hexagonal, também conhecida como arquitetura ports and adapters, é um padrão de arquitetura de software que promove a separação clara entre a lógica de negócios e os detalhes de implementação técnica. Ela facilita a modularidade, testabilidade e manutenção do sistema, ao permitir a substituição fácil de componentes externos e a reutilização de lógicas de negócio em diferentes contextos.\\
    Na wiki do repositório, há uma seção dedicada à Arquitetura que detalha a estrutura completa. Um diagrama de alto nível ilustrando essa arquitetura pode ser encontrado na figura abaixo.


\subsection{Protótipos das Telas Desenvolvidas}
  Deverão ser apresentados os links da Wiki, com uma breve descrição.

\subsection{Tecnologias Utilizadas}
  Para o desenvolvimento do frontend, utilizamos as seguintes tecnologias principais:\\
React: Uma biblioteca JavaScript amplamente usada para construir interfaces de usuário dinâmicas e responsivas.\\
Next.js: Um framework sobre React que suporta renderização no lado do servidor (SSR) e otimização de desempenho, ideal para aplicações web escaláveis.\\
JavaScript/TypeScript: Linguagens de programação essenciais para implementar a lógica da aplicação, com TypeScript oferecendo tipagem estática opcional para um desenvolvimento mais robusto e seguro.\\

Além disso, utilizamos as seguintes bibliotecas e ferramentas específicas:
Axios: Uma biblioteca para realizar requisições HTTP de forma simplificada e eficiente.\\

Keycloak: Um sistema de gerenciamento de identidade e acesso utilizado para autenticação e autorização seguras.\\
React Router (quando necessário em Next.js): Uma biblioteca para roteamento que facilita a navegação dentro de aplicações React complexas.\\

No backend, utilizamos as seguintes tecnologias:\\
Java: Uma linguagem de programação robusta e amplamente adotada para o desenvolvimento de aplicações empresariais.\\
Framework Spring Boot: Um framework Java que simplifica a criação de aplicações web e serviços RESTful, oferecendo uma configuração mínima e produtividade elevada.\\
Gerenciador de dependências Maven: Utilizado para gerenciar as dependências do projeto Java de forma eficiente e automatizada.\\
Testes com JUnit: Um framework de testes unitários para Java que facilita a implementação e execução de testes automatizados.\\

Ao acessar a wiki dentro do repositório, é possível encontrar a seção Frontend e Backend, onde é detalhada toda as tecnologias utilizadas no projeto.
