\subsection{Sprint 4}

A Sprint 4 marcou a fase final do nosso projeto, focada em desenvolver a rota para buscar todas as respostas do paciente no backend. Este objetivo representava um componente essencial para a funcionalidade de nossa API, permitindo que os dados dos pacientes fossem acessados de forma organizada e eficiente.\\

A tarefa principal da sprint foi implementar essa rota do zero. Apesar de inicialmente enfrentar dificuldades técnicas, especialmente na construção de uma base sólida para a rota, a colaboração com o AGES IV, Jhonata, no início do processo foi fundamental. Sua ajuda permitiu superar os obstáculos iniciais e me deu o impulso necessário para prosseguir de forma independente. Uma vez superados esses desafios iniciais, consegui desenvolver o restante da funcionalidade com maior confiança e eficiência.\\

Além de concluir a rota principal, dediquei parte do meu tempo para ampliar a cobertura de testes unitários, focando em algumas funcionalidades da classe ‘RespostaService’ que anteriormente estavam sem testes.\\

Um contratempo adicional durante esta sprint foi um problema técnico com meu computador, o que me obrigou a adaptar rapidamente meu plano de trabalho. Acabei que todas as tarefas que fui fazer, tinha que pedir para algum outro colega compartilhar o código comigo ao vivo, via uma extensão do VSCode, chamada ‘LiveShare’.\\

Em termos de aprendizado técnico, esta sprint foi extremamente produtiva. Aprofundei significativamente meu conhecimento em arquitetura de software e em testes unitários. Além disso, aprendi a utilizar Data Transfer Objects (DTOs) corretamente, uma habilidade crucial para a manipulação eficiente de dados entre diferentes camadas de uma aplicação.\\

Como sendo a última Sprint do projeto, sinto que tive um ótimo progresso em conhecimento técnico em desenvolvimento de software. Com certeza, estou saindo desta última Sprint com muito mais conhecimento e confiança para enfrentar outros projeto grandes, sejam pessoais ou dentro de empresas.
