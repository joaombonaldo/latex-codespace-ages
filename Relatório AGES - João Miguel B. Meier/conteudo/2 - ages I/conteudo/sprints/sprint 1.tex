\subsection{Sprint 1}

Após a apresentação dos resultados da Sprint 0 aos stakeholders, onde mostramos os fluxogramas e mockups desenvolvidos, e onde discutimos o planejamento para a Sprint 1, nossa equipe foi organizada em quatro squads. Cada squad foi composta por uma divisão equilibrada entre membros focados em backend e frontend. Eu fui alocado na squad 4, que recebeu a responsabilidade pela User Story 03, dedicada ao cadastro de pacientes.\\

Essa user story especificamente envolvia o desenvolvimento de funcionalidades para cadastrar pacientes sob a supervisão de um médico. Como ainda não havíamos implementado a funcionalidade de cadastro de médicos, utilizamos um médico de teste para integrar essa nova funcionalidade. Junto com Guilherme Ochoa, também do backend da squad 4, decidimos adotar uma abordagem colaborativa em vez de dividir tarefas especificamente. Optamos por trabalhar juntos em todos os aspectos da funcionalidade, o que nos permitiu aprender e complementar um ao outro de maneira mais eficiente.\\

Inicialmente, enfrentei desafios significativos devido à falta de experiência prática com projetos reais. Foi durante esse período que foi percebido a importância do Pair Programming, uma técnica de programação colaborativa que se provou fundamental para o meu desenvolvimento. Nas sessões de Pair Programming, realizadas virtualmente, recebi orientação intensiva do AGES IV sobre a arquitetura hexagonal utilizada no backend. Essa orientação foi crucial para entender onde e como implementar cada segmento da funcionalidade em desenvolvimento.\\

Focamos inicialmente na criação de funções robustas para a verificação de dados e tratamento de exceções. Após estabelecermos uma base sólida para a funcionalidade de cadastro, prosseguimos para a etapa de testes.\\

Em colaboração com o AGES IV, desenvolvemos testes unitários utilizando a biblioteca Mockito. A experiência de desenvolver testes unitários foi enriquecedora, proporcionando uma compreensão mais profunda das práticas de desenvolvimento de software e da importância de uma base de código testável e sustentável.\\

A Sprint 1 provou ser uma experiência intensiva e instrutiva, consolidando as habilidades técnicas da equipe e aprimorando nossa capacidade de trabalhar de forma colaborativa sob a estrutura de desenvolvimento ágil. As lições aprendidas e as habilidades desenvolvidas durante essa fase do projeto serão fundamentais para o sucesso das próximas sprints.