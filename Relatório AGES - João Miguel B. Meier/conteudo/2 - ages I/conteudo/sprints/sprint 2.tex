\subsection{Sprint 2}

A Sprint 2 começou com o foco em desenvolver e testar a funcionalidade de visualizar pacientes cadastrados sob a supervisão de um médico específico. Esta User Story foi alocada para nossa squad 4, continuando a parceria entre mim e o Guilherme Ochoa no backend. Concluímos esta funcionalidade com sucesso, seguindo o planejamento inicial e as técnicas já consolidadas de desenvolvimento ágil que havíamos adotado.\\

A implementação dessa funcionalidade envolveu a criação de endpoints específicos na API, garantindo a comunicação eficiente e segura entre o frontend e o backend. A experiência adquirida na sprint anterior, especialmente em termos de testes unitários e arquitetura de software, facilitou significativamente nossa tarefa. Com a utilização da biblioteca Mockito, desenvolvemos testes que validaram a integridade e funcionalidade dos novos endpoints de forma eficaz.\\
No entanto, a segunda parte da sprint envolveu um desafio maior com a User Story para cadastrar áreas na API administrador. A dificuldade aqui foi a necessidade de construir essa funcionalidade do zero, uma vez que a API administrador até então possuía apenas o código base sem funcionalidades específicas implementadas. Esse trabalho revelou-se mais complicado do que o antecipado, e não conseguimos finalizar a funcionalidade dentro do prazo desta sprint.\\

Outro ponto desafiador que acabou ocorrendo nessa Sprint 2, foram as enchentes que afetaram a cidade de Porto Alegre e Região Metropolitana. Com o cancelamento das aulas por duas semanas, o grupo inteiro acabou se desconectando um pouco do projeto, inclusive, acabamos perdendo alguns colegas por desistência da disciplina, o que, também, acabou desfalcando o time como um todo.\\
O reforço nos aprendizados sobre testes unitários foi um ponto positivo, visto que acabei fazendo mais testes para este endpoint de visualizar pacientes cadastrados sob a supervisão de um médico específico.\\

Para as próximas etapas do projeto, planejamos continuar o desenvolvimento da User Story de cadastrar áreas na API administrador, além de iniciar a integração CI/CD para automatizar os processos de teste e deployment. Este próximo passo é crucial para manter a consistência e a qualidade do software, permitindo uma integração contínua e entrega contínua das funcionalidades desenvolvidas.