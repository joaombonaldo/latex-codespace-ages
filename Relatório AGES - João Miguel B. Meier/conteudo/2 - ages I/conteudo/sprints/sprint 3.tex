\subsection{Sprint 3}
Durante a Sprint 3, nosso foco estava em concluir a implementação da User Story para cadastrar áreas na API Administrador e avançar nos estudos de CI/CD, com ênfase especial no Kubernetes. Esta etapa do projeto se mostrou desafiadora e instrutiva, proporcionando uma oportunidade significativa para desenvolver habilidades técnicas avançadas e aplicá-las no projeto.\\

A User Story de cadastrar áreas na API Administrador, uma continuação do trabalho iniciado na sprint anterior, não foi completada a 100 por cento. Enfrentamos dificuldades significativas devido à complexidade de implementar essa funcionalidade do zero, o que exigiu uma compreensão profunda das necessidades do sistema e uma integração eficaz com as partes já existentes da API. Apesar dos desafios, fizemos progressos consideráveis, estabelecendo uma base sólida sobre a qual poderíamos construir na conclusão do projeto.\\

Paralelamente, dediquei um tempo substancial para estudar Kubernetes, uma ferramenta essencial para a gestão de containers e automatização de CI/CD. O aprendizado adquirido foi profundamente enriquecedor, aumentando minha compreensão sobre como essas tecnologias podem ser aplicadas para melhorar a eficiência e escalabilidade das aplicações. Esse estudo não apenas ampliou meu conhecimento técnico mas também me permitiu auxiliar efetivamente outros membros da equipe.\\

Além da minha carga de trabalho principal, auxiliei o Matheus Bueno na task de dar o ‘Match’ entre os estudos e os pacientes, contribuindo na elaboração de testes e na revisão do código para o Merge Request. Esta colaboração foi uma oportunidade valiosa para aplicar as habilidades em testes que havia desenvolvido até o momento, além de fortalecer a dinâmica de equipe e o compartilhamento de conhecimentos.\\
As lições aprendidas durante esta sprint reforçaram a importância de uma base sólida em testes e a compreensão de ferramentas de CI/CD como Kubernetes. Estes
aprendizados serão cruciais não apenas para as tarefas técnicas futuras, mas também para a minha carreira como um todo.\\
Para a próxima sprint, que será a última deste projeto, foi dada prioridade a algumas tarefas específicas. Fiquei responsável por desenvolver a funcionalidade de visualizar todas as respostas do paciente no backend.
