\subsection{Sprint 0}

O projeto "Let Me Trial" começou imediatamente após uma reunião inicial comos stakeholders, que apresentaram a visão geral do projeto e as expectativas em relação ao sistema. Durante esta reunião, foi identificada a necessidade de um sistema web que facilitasse a conexão, ou "match", entre pacientes de médicos e estudos clínicos adequados. Após a apresentação, discutimos as tecnologias que seriam utilizadas e as áreas específicas de interesse de cada membro da equipe, decidindo entre backend, frontend ou fullstack.\\


Com as diretrizes claras, a equipe passou a definir as tarefas específicas para
a Sprint 0. Decidimos que as primeiras entregas seriam os fluxogramas do processo de funcionamento do software e os mockups das interfaces de usuário, utilizando o Figma como ferramenta devido à sua flexibilidade e recursos de colaboração. Cada membro da equipe desenvolveu inicialmente um fluxograma baseado na sua interpretação do sistema, que posteriormente foi discutido em grupo para consolidar um fluxograma oficial que guiaria o desenvolvimento.\\

Quanto aos mockups, distribuímos as responsabilidades de cada tela entre os membros da equipe. João, especificamente, ficou encarregado do mockup da tela de Login. Essa atividade foi essencial para visualizar a interface do usuário e garantir uma experiência consistente e intuitiva. Após a conclusão dos mockups, a equipe também discutiu e definiu um padrão de cores e estilo das interfaces, o que ajudou a manter a coesão visual do projeto.\\

Além das tarefas de design, como AGES I, foi dedicado um tempo considerável ao estudo do Java SpringBoot, a tecnologia escolhida para o desenvolvimento do backend. Realizei pequenos projetos para compreender melhor as funcionalidades da biblioteca e explorar aspectos da arquitetura de sistemas. Para reforçar meu entendimento, também revisei algumas aulas de Gerenciamento e Configuração de Software, focando em conceitos de Git, essenciais para o trabalho em equipe e controle de versões.\\

Durante esse período, enfrentei desafios adicionais propostos pelo AGES IV, incluindo um exercício prático de backend. Inicialmente, encontrei dificuldades devido à complexidade das tarefas que estavam além da minha base de conhecimento naquele momento. No entanto, com a assistência contínua dos AGES IV, através de encontros virtuais nos finais de semana, consegui superar essas barreiras e completar o desafio proposto.\\

Essa primeira Sprint foi crucial não apenas para definir a estrutura e o design do projeto, mas também para o desenvolvimento pessoal e técnico de cada membro da equipe, proporcionando uma base sólida para as próximas fases do projeto.
